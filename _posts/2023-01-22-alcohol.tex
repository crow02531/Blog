---
layout: post
title:  醇
date:   2023-01-22 14:11:42 +0800
tags: organic-chemistry
---

\documentclass{article}
\usepackage{mhchem}
\usepackage{chemfig}
\begin{document}

乙醇是最被广泛认识的醇,不是每个人都学过有机化学,但每个人都知道乙醇,也就是酒精. 人类很早就会用糖类发酵制造酒精,在中国发现的九千年前的陶器上就有酒的残留物,由此可见,新石器时代的人已经开始饮酒. 虽然酒的历史源远流长,但对乙醇结构的探索直到19世纪才开花结果,1807年N.T.Saussure确定了乙醇的化学式. 50年后A.S.Cooper发表了乙醇的结构式. 乙醇的结构是最早发现的结构之一.

\begin{center}
    \chemname{\chemfig{-[1](-[7]OH)}}{乙醇}
\end{center}

再过了约150年,于1994年2月,以Scripps研究所的K.C.Nicolaou和弗罗里达州立大学的Robert Holton领导的两个小组同时宣布完成了紫杉醇的全合成. 人们已经能够合成结构极为复杂的紫杉醇.

TODO: 紫杉醇的chemfig

\section*{醇的定义}

醇给人的第一印象是含羟基$\ce{-OH}$结构的有机物,但这其实是不准确的.

\textbf{物质}(醇)含醇羟基的有机化合物称为醇. 醇羟基是连在饱和碳上的羟基.

\textbf{注}:醇羟基中的氧一定是连在饱和碳(成4根单键的碳)上的氧,像酚、羧酸、烯醇中的羟基就不是醇羟基.

\textbf{注}:醇羟基个数为n的醇称为n元醇,二元及以上的醇统称多元醇. 含有连接n级碳的羟基的醇称n级醇,由于一、二、三级碳又称伯、仲、叔碳,一、二、三级醇也称伯、仲、叔醇. 显然不存在四级醇.

\chemname{\chemfig{-[1](-[7]OH)}}{乙醇 一元伯醇}

TODO: 香草醇的chemfig

TODO: 丙三醇的chemfig

虽然不是很常见,但可以有两个醇羟基连在同一个碳上,像这样的醇叫偕二醇.

\textbf{物质}(偕二醇)诺某一醇中的某一个碳原子上连且仅连了两个醇羟基,那么称该醇为偕二醇.

\chemname{\chemfig{C(-[4]H)(-[6]H)(-[1]OH)(-[7]OH)}}{甲二醇}

最简单的偕二醇是甲二醇$\ce{H2C(OH)2}$. 偕二醇普遍不稳定,极易脱水,但不代表它们不能大量存在. 甲醛的水溶液中几乎不存在甲醛分子,可以说甲醛分子100\%的都和水分子结合形成甲二醇. 虽然甲醛水溶液里有很多甲二醇,但没法把它们分离出来形成纯净物,在分离的过程中甲二醇很容易失水.

\textbf{物质}(邻二醇)诺某一醇中有两个连在相邻碳上的醇羟基,那么称该醇为邻二醇.

\chemname{\chemfig{(-[3])(-[5])(-[6]OH)-(-[7])(-[2])(-[6]OH)}}{频哪醇}

比方说频哪醇就是一种邻二醇.

\section*{醇的反应}

醇的反应中心是醇羟基.

TODO

\small *文章中$\ce{R}$表示第一个原子是饱和碳的基团.

醇$\ce{R-O-H}$和水$\ce{H-O-H}$具有一定的相似性,正如水是两性物质,醇也是两性的,既能电离出质子,也能接受质子.

\textbf{反应}(醇的酸性)在醇羟基中,氧的电负性比氢大,$\ce{O-H}$键的成键电子偏向氧,氢表现出一定的活性,质子能够从羟基中脱落,因而醇具有一定的酸性.

TODO

醇的酸性比水稍弱一些,可以和活泼金属反应生成氢气. 共轭碱$\ce{RO-}$碱性比$\ce{OH-}$强,在水中会水解. 醇羟基中氢的活性和氢氧键成键电子的偏移程度有关,成键电子越偏向氧,氢氧键就越容易异裂,质子就更容易释放出来,醇的酸性因而显得更强.

\textbf{例子}:往无水乙醇$\ce{C2H5OH}$中投入钠,会发生反应$\ce{2C2H5OH + 2Na -> 2C2H5ONa + H2 ^}$生成氢气和乙醇钠$\ce{C2H5ONa}$. 收集生成的$\ce{C2H5ONa}$放入水中,会立马水解生成乙醇和氢氧化钠. 如果将乙醇改为$\ce{(CF3)3COH}$,不论是与钠的反应还是之后的水解反应,都会更加剧烈.(为什么?)

除了碱性,$\ce{RO-}$还有亲核性,二者是相适应的.

\textbf{反应}($\ce{RO-}$的亲核性)$\ce{RO-}$中的氧带三对孤对电子,可以进攻带正电荷的原子或基团等正电中心,具有亲核性. $\ce{RO-}$能参与亲核取代、亲核加成等反应.

\textbf{例子}:$\ce{C2H5ONa}$能与$\ce{CH3Cl}$发生反应$\ce{C2H5O- + CH3Cl -> C2H5OCH3 + Cl-}$,这是一个SN$_2$反应. 如果体系中有碳正离子,$\ce{C2H5O-}$能直接进攻碳正离子形成碳氧键$\ce{O-C}$.

\textbf{反应}(醇的碱性)在醇羟基中,氧具有孤电子对,可以和一个$\ce{H+}$配位形成𨦡盐.

\textbf{注}:虽然氧有两个孤电子对,但配位上一个质子以后就没法再结合第二个.

TODO

就和往水里加入酸一样,水中不存在游离的质子,$\ce{H+}$会立马与水结合形成$\ce{H3+O}$. 醇也是如此,醇羟基很容易和质子结合.

\textbf{例子}:在无水乙醇$\ce{C2H5OH}$中加入$\ce{H2SO4}$,会立马形成$\ce{C2H5O+H2}$.

$\ce{RO+H2}$在水中的酸性强弱和其空间位阻有关,$\ce{R}$基的位阻越小,基团$\ce{-O^+H2}$与水形成氢键而溶剂化的程度就越大,质子更难离去,酸性就越弱.

除了能给出质子变回醇外,$\ce{RO+H2}$还有一个非常重要的性质.

\textbf{反应}($\ce{-O^+H2}$是好的离去基团)$\ce{R-O+H2}$中碳氧键$\ce{R-O}$容易异裂,产生碳正离子和水,碳正离子又常常与亲核试剂结合. 当$\ce{RO+H2}$中$\ce{R}$基空间位阻较小时,$\ce{RO+H2}$常常在亲核试剂的进攻下发生SN$_2$反应.

TODO

\textbf{例子}:叔丁醇$\ce{(CH3)3C-OH}$中加入$\ce{HCl}$,发生SN$_1$反应.

TODO

\textbf{例子}:乙醇$\ce{CH3CH2-OH}$中加入$\ce{HCl}$,发生SN$_2$反应.

TODO

有一点需要注意,虽然$\ce{R-O+H2}$中的$\ce{-O^+H2}$是一个不错的离去基团,但$\ce{ROH}$中的$\ce{-OH}$,也就是醇羟基,并不是一个好的离去基团,所以碱性环境下的醇羟基不容易被取代(碳氧键不容易断裂).

\textbf{反应}(醇的亲核性)醇羟基中的氧因为有孤电子对,可以进攻正电中心,所以带有一定的亲核性.

\textbf{注}:$\ce{ROH}$中氧的亲核性远比$\ce{RO-}$中氧的亲核性弱.

\textbf{例子}:乙醇与苯硫酰氯$\ce{C6H5SO2-Cl}$在吡啶$\ce{C5H5N}$中反应.

TODO

第一步是一个SN$_2$反应,由乙醇中的氧进攻苯磺酰氯中的硫,第二步中氧上的质子转移到作为溶剂的吡啶上,吡啶是一种常用的有机碱,其上的氮有碱性. 整个反应中乙醇的一号碳并没有发生Walden翻转.(为什么?)

上述例子中获得的产物$\ce{CH3CH2-OSO2C6H5}$有点类似$\ce{CH2CH3-O+H3}$.

\textbf{反应}($\ce{-OSO2C6H5}$是好的离去基团)$\ce{R-OSO2C6H5}$中的$\ce{-OSO2C6H5}$是好的离去基团,能在亲核试剂的进攻下以$\ce{C6H5SO2O-}$离去.

\textbf{注}:与$\ce{R-O+H3}$不同,$\ce{R-OSO2C6H5}$不容易异裂成$\ce{R+}$和$\ce{C6H5SO2O-}$,这就造成它不容易发生SN$_1$反应.

\textbf{例子}:向$\ce{CH3CH2-OSO2C6H5}$中加入碘化钠$\ce{NaI}$.

TODO

\textbf{例子}:乙醇与三甲基氯硅烷反应.

\textbf{反应}(醇的$\beta$-消除)

\textbf{反应}(醇的氧化)

\textbf{物质}(Jones试剂)将铬酐$\ce{CrO3}$溶解在稀硫酸和丙酮中,得到酸性的铬酸丙酮溶液,称该溶液为Jones试剂.

\textbf{物质}(Sarrett试剂)铬酐$\ce{CrO3}$和吡啶$\ce{C5H5N}$相混合后得到的物质称Sarrett试剂,主要成分为$\ce{(C5H5N)2\cdot CrO3}$,是一种红色的晶体状配合物.

\textbf{注}:Sarrett试剂和二氯甲烷$\ce{CH2Cl2}$一起使用效果更佳,Sarrett试剂与$\ce{CH2Cl2}$的混合物称Collins试剂.

\textbf{物质}(氯铬酸吡啶盐)氯铬酸吡啶盐简称PCC,是一种橙色的晶体,通常由吡啶加入到$\ce{CrO3}$的浓$\ce{HCl}$溶液中制得. PCC结构式如下:

TODO

\textbf{反应}(邻二醇被高碘酸氧化)

\textbf{反应}(邻二醇被四醋酸铅氧化)

\textbf{反应}(频哪醇重排)

\textbf{反应}(醇的还原)

\section*{醇的波谱}

醇羟基的红外光谱有明显特征,核磁共振氢谱由于羟基氢易于交换的缘故,需要特殊考虑.

\textbf{性质}(醇的IR)

\textbf{性质}(醇的$^1$HNMR)

\end{document}