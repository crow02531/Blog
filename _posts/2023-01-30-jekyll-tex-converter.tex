---
layout: post
title:  jekyll-tex-converter发布
date:   2023-01-30 16:12:27 +0800
tags: web coding
---

\documentclass{article}
\usepackage{ctex}
\usepackage{hyperref}
\usepackage{ulem}
\usepackage{chemfig}
\begin{document}
\sout{为了在Jekyll上不引用外部图片展示有机分子的结构式,画反应机理,煞费苦心。不论是MathJax还是KaTeX都不支持chemfig包,以后估计也不支持,果断弃用。目前没有一个关于latex的js库支持chemfig包(原因无一例外都是chemfig包依赖tikz包,而tikz太大),只能回避不适用chemfig了。接着发现了Kekule.js和smilesDrawer,这两个js库能在客户端(阅览器)上画分子结构,都支持SMILES,然而SMILES并不能表示机理(虽然有扩展过的SMILES语言支持画反应),除了SMILES之外的其他分子结构表达语言例如CML、MOL等均不支持画机理,再次碰壁。果然没chemfig包很麻烦,回避不了,也没能耐让MathJax等支持chemfig,难道只能妥协引用外部图片了吗? 不,坚决拒绝,实物照片外部引用就算了,结构式、机理等要是引用外部图片的画管理起来会烦死的,道理就跟没MathJax纯用图片写数学密集的文章一样,那是地狱!是地狱!终于,转机来了。'.tex' to '.html',\href{https://www.ctan.org/pkg/tex4ht}{tex4ht},Jekyll是静态网站生成器,只要搞个converter把Jekyll中用latex写的文章生成合适的html就行了,装了latex的电脑都有\href{https://www.ctan.org/pkg/make4ht}{make4ht},能将latex转html,而且只要你装了chemfig包,make4ht就支持带chemfig的latex转html。写个Jekyll插件,调用本地tex distribution中的make4ht做tex to html,事情不就解决了吗,管你写的latex再复杂,用了再多的包,都没有关系。}

于是花了几天的时间写了个Jekyll插件\href{https://github.com/crow02531/jekyll-tex-converter}{jekyll-tex-converter}。

这张post就是用\LaTeX{}写的,看,它支持chemfig,多好。

\chemfig{*6((=O)-N(-H)-(*5(-N=-N(-H)-))=-(=O)-N(-H)-)}
\end{document}